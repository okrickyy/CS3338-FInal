\documentclass[12pt]{article}

\usepackage[margin=1in]{geometry}
\usepackage{tabularx}
\usepackage{setspace}
\usepackage{enumitem}

\title{AstroStreet AR\\Snapshot Summary Document}
\author{Group 2: Royce Jamerson, Khalid Jamil, Michael Lieng, Ricardo Ibanez}
\date{December 2025}

\begin{document}
\maketitle

\onehalfspacing

\section{Introduction}

This document summarizes the evolution of \textit{AstroStreet AR} across four
project snapshots. Each snapshot corresponds to a different stage in the
imagined 6--12 month development timeline and shows how the scope, design, and
tooling matured over time.

\begin{center}
\begin{tabularx}{\textwidth}{|c|X|}
\hline
\textbf{Snapshot} & \textbf{Focus} \\
\hline
Start (Snapshot 1) & Define core concept, basic AR gameplay loop, and initial documentation. \\
\hline
Checkpoint 1 (Snapshot 2) & Add major features (high scores, settings), refine architecture and interfaces. \\
\hline
Checkpoint 2 (Snapshot 3) & Deepen design (player stats, difficulty, accessibility, performance considerations). \\
\hline
Final / Due Date (Snapshot 4) & Polish implementation, stabilize features, and document future work. \\
\hline
\end{tabularx}
\end{center}

%-----------------------------------------------------------
\section{Snapshot 1: Project Start}

\subsection{Snapshot Objective}

The objective of Snapshot 1 was to establish the foundation of
\textit{AstroStreet AR}:

\begin{itemize}[leftmargin=*]
    \item Define the core game concept: a mobile AR shooter where players aim
    and shoot virtual asteroids and alien ships overlaid on the real world.
    \item Implement a minimal vertical slice:
    \begin{itemize}
        \item Camera feed as background.
        \item Basic AR tracking using the chosen framework (e.g., ARCore/ARKit
        via Unity).
        \item Simple enemy objects appearing in front of the player.
        \item Basic projectile firing using tap input.
    \end{itemize}
    \item Create initial project documentation and tool setup.
\end{itemize}

\subsection{Features Implemented}

At this stage, the following features were in place:

\begin{itemize}[leftmargin=*]
    \item Basic AR gameplay loop: camera feed, simple enemies, tap-to-shoot.
    \item Initial HUD text displaying at least a score counter placeholder.
    \item Single-player, local-only gameplay (no persistence yet).
\end{itemize}

\subsection{Documentation and Tooling}

\begin{itemize}[leftmargin=*]
    \item \textbf{SDD v1.0}: Introduced the high-level architecture and main
    components for \textit{AstroStreet AR}.
    \item \textbf{SRS v1.0}: Captured the initial functional requirements and
    external interface expectations.
    \item \textbf{README / User Manual (initial)}:
    \begin{itemize}
        \item Brief project description.
        \item Very basic instructions on running or building the project.
    \end{itemize}
    \item \textbf{Jira}: Project and board created with initial tasks for core
    AR setup and minimal gameplay.
\end{itemize}

\subsection{Risks and Open Issues}

\begin{itemize}[leftmargin=*]
    \item AR tracking behavior and performance on different devices were not
    fully understood.
    \item No persistence of high scores or settings.
    \item No TestRail test runs defined yet; testing was informal.
\end{itemize}

%-----------------------------------------------------------
\section{Snapshot 2: First Checkpoint}

\subsection{Snapshot Objective}

The objective of Snapshot 2 was to add a new major feature set and refine the
requirements and design:

\begin{itemize}[leftmargin=*]
    \item Introduce a basic high-score system and dedicated High Scores screen.
    \item Add a Settings screen for simple configuration options (e.g., sound,
    sensitivity).
    \item Improve handling of AR session lifecycle and error conditions.
    \item Begin formal testing using TestRail.
\end{itemize}

\subsection{Features Implemented}

\begin{itemize}[leftmargin=*]
    \item \textbf{High scores}: Ability to save and display multiple high-score
    entries locally.
    \item \textbf{Settings}: Simple toggles or values for sound and sensitivity.
    \item \textbf{AR lifecycle handling}:
    \begin{itemize}
        \item Basic detection of tracking loss.
        \item User messages when AR is not ready or permissions are denied.
    \end{itemize}
\end{itemize}

\subsection{Documentation and Tooling Updates}

\begin{itemize}[leftmargin=*]
    \item \textbf{SDD v2.0}:
    \begin{itemize}
        \item Added detailed component responsibilities.
        \item Documented AR session lifecycle and error handling.
        \item Described key runtime scenarios (e.g., starting a session,
        firing and scoring).
        \item Refined database design with entities such as
        \texttt{HighScore}, \texttt{PlayerStats}, and \texttt{Settings}.
    \end{itemize}
    \item \textbf{SRS v2.0}:
    \begin{itemize}
        \item Added assumptions and dependencies.
        \item Added user characteristics.
        \item Expanded hardware and software interface requirements.
        \item Extended legal and ethical considerations and glossary entries.
    \end{itemize}
    \item \textbf{README}:
    \begin{itemize}
        \item Updated to mention high scores and settings.
        \item Included clearer instructions on basic gameplay.
    \end{itemize}
    \item \textbf{TestRail}:
    \begin{itemize}
        \item First Test Run created for core flows (start game, basic
        shooting, basic AR initialization).
        \item Snapshot 2 summary exported and stored in the repository.
    \end{itemize}
\end{itemize}

\subsection{Risks and Open Issues}

\begin{itemize}[leftmargin=*]
    \item High-score persistence needed more thorough testing across sessions.
    \item The difficulty curve and enemy spawn patterns were still simple and
    might not be engaging for longer-term play.
\end{itemize}

%-----------------------------------------------------------
\section{Snapshot 3: Second Checkpoint}

\subsection{Snapshot Objective}

The objective of Snapshot 3 was to deepen the design and improve game feel,
especially around difficulty and player feedback:

\begin{itemize}[leftmargin=*]
    \item Introduce aggregate player statistics to support future tuning.
    \item Refine difficulty progression and spawn limits.
    \item Incorporate accessibility and performance considerations into both
    requirements and design.
\end{itemize}

\subsection{Features Implemented}

\begin{itemize}[leftmargin=*]
    \item \textbf{Player statistics}: Tracking of total games played, best
    score, shots fired, hits, and total play time.
    \item \textbf{Improved enemy and spawn behavior}: More structured spawn
    limits and simple difficulty scaling.
    \item \textbf{Basic performance tuning}: Caps on enemies and projectiles,
    consideration for object reuse.
\end{itemize}

\subsection{Documentation and Tooling Updates}

\begin{itemize}[leftmargin=*]
    \item \textbf{SDD v3.0 (design-focused changes introduced by this stage)}:
    \begin{itemize}
        \item Game state management model (MainMenu, InitializingAR, Playing,
        Paused, GameOver).
        \item Performance and resource considerations for mobile AR.
        \item UI navigation flow between core screens.
    \end{itemize}
    \item \textbf{SRS v3.0 (requirements-focused changes introduced by this stage)}:
    \begin{itemize}
        \item Non-functional requirements for performance, usability,
        reliability, and safety.
        \item Future requirements and explicit out-of-scope features.
    \end{itemize}
    \item \textbf{README}:
    \begin{itemize}
        \item Updated to reflect more polished gameplay and design.
        \item Documented the evolution of SDD/SRS versions.
    \end{itemize}
    \item \textbf{TestRail}:
    \begin{itemize}
        \item Second Test Run added for new behavior (player stats, improved
        difficulty, updated UI flow).
        \item Snapshot 3 summary exported and stored.
    \end{itemize}
\end{itemize}

\subsection{Risks and Open Issues}

\begin{itemize}[leftmargin=*]
    \item Some accessibility enhancements (e.g., fully configurable HUD sizes)
    remained future work rather than implemented features.
    \item Online or networked features were still intentionally out of scope.
\end{itemize}

%-----------------------------------------------------------
\section{Snapshot 4: Final / Due Date}

\subsection{Snapshot Objective}

The objective of Snapshot 4 was to stabilize the project for final submission
and clearly document remaining future work:

\begin{itemize}[leftmargin=*]
    \item Polish core gameplay loop, UI readability, and AR feedback.
    \item Ensure that SDD and SRS were consistent with the implemented
    features.
    \item Finalize non-functional requirements, legal/ethical considerations,
    and future enhancements.
\end{itemize}

\subsection{Features and Polish Completed}

\begin{itemize}[leftmargin=*]
    \item Small UI refinements (text clarity, button placement, HUD polish).
    \item Better messaging when AR tracking is unstable or unavailable.
    \item Cleanup and tuning of enemy caps, spawn timing, and basic difficulty.
\end{itemize}

\subsection{Documentation and Tooling Finalization}

\begin{itemize}[leftmargin=*]
    \item \textbf{SDD v3.0}: Final design document, incorporating all previous
    changes plus final clarifications on architecture, UI flow, database
    design, and extensibility.
    \item \textbf{SRS v3.0}: Final requirements document, including
    non-functional requirements and an explicit list of future/out-of-scope
    features (e.g., online leaderboards, new modes).
    \item \textbf{README / User Manual}:
    \begin{itemize}
        \item Final description of features and how to run the project.
        \item Links to Jira board and notes about TestRail runs and summaries.
        \item Section describing documentation evolution across versions.
    \end{itemize}
    \item \textbf{TestRail}:
    \begin{itemize}
        \item Final Test Run created to cover the stable feature set near the
        due date.
        \item Snapshot 4 summary exported and stored in the repository.
    \end{itemize}
\end{itemize}

\subsection{Future Work}

Although the current version completes the course objectives, the following
areas are identified as promising future work:

\begin{itemize}[leftmargin=*]
    \item Online leaderboards and cloud-synced player profiles.
    \item Additional game modes (timed, endless, cooperative).
    \item More varied enemy types and behaviors.
    \item Expanded accessibility options (configurable HUD, colorblind modes,
    haptic feedback settings).
    \item Potential support for additional AR platforms and devices.
\end{itemize}

\end{document}
