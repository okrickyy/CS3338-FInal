\documentclass[12pt]{article}

% Page layout
\usepackage{geometry}
\geometry{
  letterpaper,
  margin=1in
}

% Fonts and formatting
\usepackage{lmodern}
\usepackage{setspace}
\usepackage{xcolor}

% For better title control
\usepackage{titling}
\usepackage{tabularx}


\title{\vspace{2cm}
  \textbf{\Huge AstroStreet AR}\\[1em]
  \textbf{\Large Software Requirements Specification (SRS)}\\[0.5em]
  \rule{0.8\textwidth}{0.5pt}\\[0.5em]
  \large CS 3338 -- Software Engineering Tools
}
\author{
  \large Group 2\\[0.5em]
  Royce Jamerson\\
  Khalid Jamil\\
  Michael Lieng\\
  Ricardo Ibanez
}
\date{\large December 2, 2025}

\begin{document}

% ---------- Cover Page ----------
\begin{titlepage}
    \centering
    \vspace*{2cm}

    {\Huge \textbf{AstroStreet AR}\par}
    \vspace{0.8cm}
    {\Large \textbf{Software Requirements Specification (SRS)}\par}
    \vspace{0.5cm}
    {\large Assignment 14\par}
    \vspace{0.5cm}
    {\large CS 3338 -- Software Engineering Tools\par}
    \vspace{0.5cm}
    {\large California State University, Los Angeles\par}

    \vfill

    {\large \textbf{Group 2}\par}
    \vspace{0.4cm}
    {\large Royce Jamerson\par}
    {\large Khalid Jamil\par}
    {\large Michael Lieng\par}
    {\large Ricardo Ibanez\par}

    \vfill

    {\large Date: December 2, 2025\par}

    \vspace*{1.5cm}
\end{titlepage}

% ---------- Table of Contents ----------
\pagenumbering{roman}        % Roman numerals for front matter
\setcounter{tocdepth}{2}     % Include sections and subsections
\tableofcontents
\newpage

\pagenumbering{arabic}       % Start normal page numbers for body

% ---------- SRS Sections ----------

% --- 1. Version Description
\section{Version Description}

This section tracks the revision history of the
\textit{AstroStreet AR} Software Requirements Specification (SRS)
for Group 2 in CS 3338 -- Software Engineering Tools.

\begin{center}
\begin{tabularx}{\textwidth}{|c|X|c|}
\hline
\textbf{Version} & \textbf{Description} & \textbf{Date} \\
\hline
1.0 &
Initial version of the Software Requirements Specification for
\textit{AstroStreet AR}, prepared for Assignment 14. &
December 2, 2025 \\
\hline
\end{tabularx}
\end{center}

Future revisions of this document can be recorded by adding new
rows to the table with updated version numbers, dates, and brief
descriptions of the changes.

% --- 2. Introduction
\section{Introduction}

\subsection{Purpose}

The purpose of this Software Requirements Specification (SRS) is to
describe the functional and non-functional requirements of
\textit{AstroStreet AR}, a mobile augmented reality shooter game
developed by Group 2 for CS 3338 -- Software Engineering Tools at
California State University, Los Angeles.
This document defines what the system shall do from the user's and
stakeholders' perspectives, and serves as a contract between the
development team and the project stakeholders.

\subsection{Scope}

\textit{AstroStreet AR} is a smartphone game in which players go
outside, point their phone at the sky or surrounding environment, and
see virtual asteroids and alien ships overlaid on the live camera view.
The player rotates their body and phone to aim and taps the screen to
shoot incoming targets.
The system tracks score, player health or shields, and increasing
difficulty as enemies appear over time.

The main goals of the system are to:
\begin{itemize}
    \item Provide an accessible AR game experience that can be played
    outdoors using a standard smartphone.
    \item Offer simple, intuitive controls suitable for quick play
    sessions.
    \item Track and persist player high scores and basic settings.
\end{itemize}

This SRS covers the requirements for the initial single-player version
of \textit{AstroStreet AR}.
Future enhancements, such as online multiplayer modes or cloud-based
leaderboards, are considered out of scope for this document and may be
addressed in later revisions.

\subsection{Intended Audience}

This document is intended for the following audiences:
\begin{itemize}
    \item \textbf{Developers}: to understand the required behavior of
    the system and implement it accordingly.
    \item \textbf{Testers}: to design test cases that verify whether
    the implemented system satisfies the stated requirements.
    \item \textbf{Course Instructor and Teaching Staff}: to evaluate the
    completeness and correctness of the requirements as part of
    Assignment 14.
    \item \textbf{Future Maintainers}: to use as a reference when
    modifying or extending the system.
\end{itemize}

\subsection{Document Overview}

The remainder of this SRS is organized as follows:
\begin{itemize}
    \item \textbf{Version Description} summarizes the revision history
    of this document.
    \item \textbf{External Interface Requirements} describes how users
    and external systems interact with \textit{AstroStreet AR}, including
    user interface behavior and dependencies on hardware and software
    platforms.
    \item \textbf{Legal and Ethical Considerations} identifies safety,
    privacy, and ethical concerns related to using an AR game in
    outdoor environments and explains how the system addresses them.
    \item \textbf{Glossary} defines key terms and acronyms used
    throughout the SRS.
\end{itemize}

% --- 3. External Interface Requirements
\section{External Interface Requirements}

This section describes how users and external systems interact with
\textit{AstroStreet AR}.
It specifies requirements for the user interface, hardware, software,
and communication interfaces.

\subsection{User Interface}

\subsubsection{Main Menu}

\begin{itemize}
    \item \textbf{UI-1:} The system shall display a main menu when the
    application is launched.
    \item \textbf{UI-2:} The main menu shall display the game title
    \textit{AstroStreet AR} and the following buttons:
    \begin{itemize}
        \item \textbf{Start Game}
        \item \textbf{High Scores}
        \item \textbf{Settings}
        \item \textbf{Exit} (if supported by the platform)
    \end{itemize}
    \item \textbf{UI-3:} When the user taps \textbf{Start Game}, the
    system shall start a new AR gameplay session.
    \item \textbf{UI-4:} When the user taps \textbf{High Scores}, the
    system shall display the high scores screen.
    \item \textbf{UI-5:} When the user taps \textbf{Settings}, the
    system shall display the settings screen.
\end{itemize}

\subsubsection{AR Gameplay Screen}

\begin{itemize}
    \item \textbf{UI-6:} During gameplay, the system shall display the
    live camera feed as the background of the screen.
    \item \textbf{UI-7:} The system shall display a crosshair or reticle
    centered on the screen to indicate the aiming direction.
    \item \textbf{UI-8:} The system shall display the player's current
    score in a corner of the screen (e.g., upper-left).
    \item \textbf{UI-9:} The system shall display the player's current
    health or shield as a bar or similar indicator (e.g., in the
    upper-right).
    \item \textbf{UI-10:} If a time limit is used, the system shall
    display the remaining time during gameplay.
    \item \textbf{UI-11:} The system shall provide a pause button that
    the user can tap to open the pause menu.
    \item \textbf{UI-12:} When the user taps the screen within the AR
    gameplay view, the system shall treat this as a ``shoot'' action
    and fire a projectile in the current aiming direction.
\end{itemize}

\subsubsection{Pause Menu}

\begin{itemize}
    \item \textbf{UI-13:} When the pause button is tapped, the system
    shall pause gameplay and display a pause menu overlay.
    \item \textbf{UI-14:} The pause menu shall provide options to:
    \begin{itemize}
        \item \textbf{Resume} gameplay.
        \item Open \textbf{Settings} (limited to in-game options).
        \item \textbf{Quit to Main Menu}.
    \end{itemize}
\end{itemize}

\subsubsection{Game Over Screen}

\begin{itemize}
    \item \textbf{UI-15:} When the game session ends (e.g., the player's
    health reaches zero or a time limit expires), the system shall
    display a game over screen.
    \item \textbf{UI-16:} The game over screen shall display the
    player's final score.
    \item \textbf{UI-17:} If the final score is a new high score, the
    system shall indicate this to the user (e.g., with a message such
    as ``New High Score!'').
    \item \textbf{UI-18:} The game over screen shall provide options to
    \textbf{Play Again} and \textbf{Return to Main Menu}.
\end{itemize}

\subsubsection{High Scores Screen}

\begin{itemize}
    \item \textbf{UI-19:} The system shall provide a high scores screen
    that displays a list of previously recorded scores.
    \item \textbf{UI-20:} The high scores screen shall show, at a
    minimum, the score value and the date each score was achieved.
    \item \textbf{UI-21:} The high scores screen shall provide a way for
    the user to return to the main menu.
\end{itemize}

\subsubsection{Settings Screen}

\begin{itemize}
    \item \textbf{UI-22:} The system shall provide a settings screen
    accessible from both the main menu and the pause menu.
    \item \textbf{UI-23:} The settings screen shall allow the user to
    enable or disable sound effects.
    \item \textbf{UI-24:} The settings screen shall allow the user to
    adjust control sensitivity for aiming.
    \item \textbf{UI-25:} The system shall persist settings changes so
    that they are applied the next time the application is launched.
\end{itemize}

\subsection{Hardware Interfaces}

\begin{itemize}
    \item \textbf{HW-1:} The system shall run on a smartphone device
    that includes a rear-facing camera.
    \item \textbf{HW-2:} The system shall require device motion sensors
    (such as a gyroscope and accelerometer) to track the orientation of
    the phone for AR gameplay.
    \item \textbf{HW-3:} The system shall require a touchscreen to
    receive tap input for shooting and UI navigation.
    \item \textbf{HW-4:} The system should be playable while the user is
    holding the device in portrait or landscape orientation (the chosen
    orientation may be fixed by design).
\end{itemize}

\subsection{Software Interfaces}

\begin{itemize}
    \item \textbf{SW-1:} The system shall run on a supported mobile
    operating system, such as Android or iOS, that can host the chosen
    game engine (e.g., Unity).
    \item \textbf{SW-2:} The system shall use an augmented reality
    framework (for example, ARCore on Android or ARKit on iOS) to
    obtain camera images and device pose information.
    \item \textbf{SW-3:} The system shall use the mobile operating
    system's storage APIs or the game engine's persistence features to
    save and load high scores and settings.
    \item \textbf{SW-4:} If a game engine such as Unity is used, the
    system shall interact with engine components through their
    documented APIs.
\end{itemize}

\subsection{Communication Interfaces}

For the initial version of \textit{AstroStreet AR}, no network-based
communication is strictly required.

\begin{itemize}
    \item \textbf{COM-1:} The system shall be able to run entirely
    offline, without requiring a network connection for core gameplay.
    \item \textbf{COM-2:} If future versions introduce an online
    leaderboard or cloud storage, the system shall communicate with the
    remote service using secure HTTP-based APIs (e.g., HTTPS REST
    endpoints). Such features are considered out of scope for the
    current SRS but may be added in future revisions.
\end{itemize}

% --- 4. Legal and Ethical Considerations
\section{Legal and Ethical Considerations}

This section identifies legal, safety, and ethical issues related to
the use of \textit{AstroStreet AR}, and states requirements to help
mitigate these risks.

\subsection{Physical Safety in Outdoor Environments}

Because \textit{AstroStreet AR} encourages players to move around
outdoors while looking at their phone screen, there is a risk that
players may become distracted and fail to notice real-world hazards
such as vehicles, obstacles, or uneven ground.

\begin{itemize}
    \item \textbf{SAFE-1:} The system shall display a safety warning
    before the start of gameplay, reminding users to stay aware of their
    surroundings and to avoid playing while crossing streets, driving,
    or operating vehicles.
    \item \textbf{SAFE-2:} The system shall allow the user to dismiss
    the safety warning only by acknowledging it (for example, by tapping
    an ``I Understand'' or similar button).
    \item \textbf{SAFE-3:} The system shall pause gameplay when the
    application loses focus or is minimized, so that the user is not
    encouraged to continue playing while switching apps or locking the
    device.
\end{itemize}

These measures are intended to reduce, but cannot eliminate, the risk of
accidents while playing the game.

\subsection{Privacy and Use of Device Sensors}

\textit{AstroStreet AR} accesses the device camera and motion sensors
to provide an AR experience.
In some implementations, it may also store limited gameplay data such
as high scores.

\begin{itemize}
    \item \textbf{PRIV-1:} The system shall request permission to use
    the device camera in accordance with mobile platform guidelines.
    Gameplay shall not begin until camera access is granted or the user
    exits the game.
    \item \textbf{PRIV-2:} The system shall use the camera feed only to
    render the AR gameplay experience and shall not transmit camera
    images or video frames to external servers in the initial version.
    \item \textbf{PRIV-3:} The system shall not collect personally
    identifiable information (such as full name, address, or precise
    location), beyond optional player initials or a nickname for
    high-score display.
    \item \textbf{PRIV-4:} Any data stored for high scores or settings
    shall be kept locally on the device in the initial version.
\end{itemize}

If future versions introduce online features (such as cloud-based
leaderboards), additional privacy requirements and notices shall be
defined in a later revision of this SRS.

\subsection{Data Usage and Retention}

The game stores a small amount of persistent data (high scores and
settings) to improve the user experience.

\begin{itemize}
    \item \textbf{DATA-1:} The system shall store only the minimum data
    necessary for its functionality, such as scores, dates, and simple
    configuration settings.
    \item \textbf{DATA-2:} The system shall not store sensitive data,
    such as passwords, payment information, or contact lists, as part of
    normal gameplay.
    \item \textbf{DATA-3:} The system shall allow stored high scores and
    settings to be deleted by uninstalling the application, which
    removes local data from the device.
\end{itemize}

\subsection{Age-Appropriate Use}

\textit{AstroStreet AR} is intended as a casual arcade-style game with
non-realistic depictions of asteroids, alien ships, and explosions.

\begin{itemize}
    \item \textbf{AGE-1:} The system shall avoid realistic depictions of
    physical harm to real people or animals.
    \item \textbf{AGE-2:} The system shall be designed to be appropriate
    for a general audience and shall not include explicit language or
    graphic content.
    \item \textbf{AGE-3:} If the game is distributed through a mobile
    app store, the stated age rating shall be consistent with the
    content (for example, ``Everyone'' or an equivalent rating), subject
    to the store's guidelines.
\end{itemize}

\subsection{Ethical Use of Augmented Reality}

AR applications can influence how users perceive their surroundings and
interact with public spaces.

\begin{itemize}
    \item \textbf{ETH-1:} The system shall not encourage users to enter
    restricted, private, or dangerous areas in order to play the game.
    \item \textbf{ETH-2:} The system shall not place game objectives or
    incentives specifically in locations that could increase the risk
    of harm (such as the middle of streets or near moving vehicles).
    \item \textbf{ETH-3:} The system shall present AR content in a way
    that respects public spaces and does not promote harassment or
    nuisance behavior toward other people.
\end{itemize}

By addressing these legal and ethical considerations, the project aims
to provide an enjoyable AR experience while minimizing potential safety,
privacy, and ethical concerns for players and bystanders.


% --- 5. Glossary 
\section{Glossary}

This section defines key terms and acronyms used throughout the
\textit{AstroStreet AR} Software Requirements Specification.

\begin{center}
\begin{tabularx}{\textwidth}{|c|X|}
\hline
\textbf{Term} & \textbf{Definition} \\
\hline
AR (Augmented Reality) &
A technology that overlays virtual objects onto a live view of the
real world, typically using the device camera and motion sensors. \\
\hline
UI (User Interface) &
The visual elements and controls that the player interacts with,
such as buttons, menus, and icons. \\
\hline
HUD (Heads-Up Display) &
On-screen information shown during gameplay, including score, health
or shield, timer, and other status indicators. \\
\hline
Game Session &
A single continuous playthrough that begins when the player starts a
game and ends when the player loses, quits, or returns to the main
menu. \\
\hline
High Score &
A recorded score from a completed game session that is stored and
displayed on the High Scores screen. \\
\hline
Persistent Data &
Information that is saved to device storage so it remains available
after the application is closed, such as high scores and settings. \\
\hline
Camera Feed &
The live video stream captured by the device's camera and used as
the background for the AR gameplay view. \\
\hline
Sensitivity (Control Sensitivity) &
A configurable setting that controls how quickly the aiming reticle
responds to device movement. \\
\hline
Safety Warning &
A message shown to the user reminding them to stay aware of their
surroundings and avoid dangerous behavior while playing the game. \\
\hline
Mobile Platform &
The underlying operating system and environment on which the game
runs, such as Android or iOS. \\
\hline
\end{tabularx}
\end{center}



\end{document}
